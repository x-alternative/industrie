\documentclass[a4paper]{article}
\usepackage[utf8]{inputenc}
\usepackage[T1]{fontenc}
\usepackage{hyperref}
\usepackage{graphicx}
\usepackage{float}
\usepackage{svg}
\usepackage{caption}
\captionsetup{font=footnotesize}
\captionsetup{width=0.8\textwidth}
\usepackage[francais]{babel}
\author{Michel Donnedu \& Jean-Charles Hourcade\\
\large \textit{X-Alternative}}
\date{\today}
\title{Une ambition pour l'industrie française}
\renewcommand{\contentsname}{Table des matières}
\begin{document}

\maketitle
\newpage
\tableofcontents
\newpage
\section{Avertissement}
Cette note est un document de travail pour le groupe de travail « Industrie, énergie, mobilités ». Elle vise la préparation d’une expression de X-Alternative sur l‘industrie nationale. Se fondant essentiellement sur des traitements des données de la comptabilité nationale publiées par l’INSEE, apportant des premiers commentaires, elle est pour l’heure critiquable, amendable, incomplète et inachevée. Il y manque notamment des analyses sur l’état actuel des branches industrielles et  les perspectives à construire. Il manque aussi des illustrations de situations d’entreprises. La présentation des enjeux en introduction mérite aussi d’être complétée et enrichie. La présente note se veut donc une invitation à la contribution des membres du groupe de travail. 
\newpage

La situation de l’industrie française était déjà extrêmement préoccupante. Démantèlements, cessions, désinvestissement et politiques publiques insuffisantes ou inadaptées ont fait régresser les capacités industrielles françaises. 
Ce constat est apparu violemment dans le contexte de la pandémie. L’incapacité à produire ou importer des équipements hospitaliers essentiels a aggravé la crise sanitaire. Le sous-équipement des personnels hospitaliers les a mis en danger. L’absence de tests et la saturation du système de santé ne nous ont laissé que le confinement général comme moyen de protéger les populations. 
Le confinement général se mesurera bientôt en faillites et chômage, affaiblissant d’autant plus un tissu économique et industriel déjà fragilisé. Par ailleurs, la politique déflationniste envisagée par le gouvernement visant à renforcer la compétitivité des entreprises en baissant les salaires risque au contraire de déprimer la demande intérieure, aujourd’hui tournée pour près de moitié vers les produits industriels. 
Ce qui risque de se jouer, en plus d’une catastrophe sociale, c’est un affaiblissement supplémentaire de nos capacités industrielles. Dans certains secteurs, le choc risque d’être létal et nécessite une réponse d’ampleur. Continuer de laisser mourir ou délocaliser trop l’industrie nous exposerait à une vulnérabilité encore accrue en cas de nouvelle pandémie ou de catastrophe climatique. 
Cette note propose un tableau de la désindustrialisation en France depuis les années 1970, une évaluation de ses conséquences économiques et budgétaires ainsi que des propositions pour remédier à la situation, dans le contexte d’une rupture écologique nécessaire. 



\section{Enjeux}
\subsection{Les transitions énergétiques et écologiques}
Avec les impératifs de la lutte contre les dérèglements climatiques et la dégradation accélérée de l’écosystème, l’industrie doit entamer une transition. La conscience en est répandue et la crise sanitaire liée à l’épidémie du Covid 19 l’a encore fait grandir. Mais la transition peine à se mettre en place, du moins au rythme voulu par les exigences écologiques et les attentes sociales. Pourtant l’industrie doit s’atteler à la grande cause de la réduction de l’empreinte des activités humaines sur la planète : économie des ressources terrestres, absence de rejets de gaz à effet de serre, dépollution des rejets. Ces objectifs, vitaux pour l’humanité, interpellent l’ensemble de la chaîne de production : la conception des produits, dont la durabilité et les aptitudes à être réparés et recyclés doivent devenir la boussole de l’ingénierie, en bannissant toute notion d’obsolescence programmée ; la fabrication ou l’élaboration des produits, qui doit minimiser les consommations d’énergie et viser l’absence de rejet de gaz à effet de serre ; la récupération des déchets et la dépollution des effluents. Ces objectifs seront efficacement poursuivis non pas par la décroissance des activités industrielles, mais par la construction d’une industrie transformée. Elle doit en effet produire les matériaux, machines et équipements permettant d’assurer les transitions énergétiques et écologiques.  La réduction des intensités énergétiques nécessite des productions croissantes ou nouvelles en matière de matériaux de construction et d’isolation, de machines et d’appareils sobres. La décarbonation de  la production d’énergie  nécessite la croissance des industries  produisant les centrales nucléaires, les éoliennes, panneaux solaires et leurs équipements.  L’économie des ressources terrestres  nécessite le développement et la création d’activités de réparation et de recyclage des produits, de traitement des déchets, de dépollution des effluents…

Ces transitions de l’industrie sont à accomplir en France dans une situation d’affaiblissement de son industrie, qui est devenue structurellement déficitaire au regard des besoins actuels de la société et de l’économie en matière de produits. Cette situation est loin d’être neutre au regard des enjeux climatiques. L’explosion des échanges et donc des transports de marchandises depuis l’autre bout du monde implique des émissions de gaz à effet de serre importantes qui vont croissant. De plus, les stratégies de délocalisation, via les importations, se défaussent hypocritement de toute responsabilité sur les conditions de production de ces biens.

Depuis le déclin du charbon et  sa substitution par le pétrole et le gaz naturel, les 2/3 du déficit industriel de la France proviennent de l’importation des ressources terrestres (combustibles fossiles et minéraux). Traditionnellement, on pouvait penser que le reste de l’industrie devait dégager les excédents permettant de compenser « la facture pétrolière ». Cela n’étant plus réalisé depuis 20 ans, retrouver un équilibre des échanges devrait être l’un des objectifs prioritaires de toute politique industrielle digne de ce nom. Mais il ne s’agit plus de tenter d’exporter toujours plus pour payer les importations de pétrole et de gaz, il s’agit que la génération à venir puisse réduire progressivement de façon drastique sa dépendance aux hydrocarbures, et de commencer dès aujourd’hui à faire le nécessaire.


\subsection{Une industrie pour l’indépendance nationale}
A l’heure de l’intégration européenne et de la mondialisation, on (re)découvre que la notion d’indépendance nationale a conservé tout son sens. Les carences et insuffisances observées à l’occasion de la crise sanitaire du Covid-19 ont rappelé l’importance de pouvoir disposer sur le territoire national des capacités industrielles garantissant la fourniture des produits vitaux pour la nation. 

Mais leur localisation ne suffit pas, la question de leur contrôle se pose avec autant de force. On avait vu avant la crise que la prise de contrôle de la branche énergie d’Alstom par General Electric pouvait conduire à  ce qu’un siège social outre-Atlantique puisse décider d’abandonner des activités de fabrication et de maintenance essentielles à la filière nucléaire française. On a vu pendant la crise qu’ayant repris l’usine de fabrication de masques de Paintel en Bretagne, Honeywell avait en 2018 délocalisé la production en Tunisie, après avoir fait détruire les 8 machines capables de produire 20 millions de masques par mois. Et en sortie de période de confinement, on apprend qu’à nouveau General Electric, profitant des mesures gouvernementales allégeant « provisoirement » les procédures de consultation des élus du personnel au nom de la crise économique, lance une nouvelle délocalisation de la fabrication et de la maintenance des turbines à gaz et ajoute à Belfort 400 nouvelles suppressions d’emplois à celles de 2019.

La propriété des industries stratégiques est un véritable enjeu pour l’indépendance du pays. Cela mérite que soient totalement repensés la nature et les moyens de l’intervention publique dans le champ économique, et tout particulièrement sa composante industrielle.  Cet enjeu appelle lui aussi à rééquilibrer le commerce extérieur  des biens industriels. En effet, in fine le pays règle son déficit courant en vendant le capital de ses entreprises. Une part sensible de la capitalisation des groupes côtés en France est dans les mains des grands fonds d’investissement américains. Seule une intervention publique volontariste peut inverser ce cercle vicieux.

\subsection{Enjeux sociaux et sur le travail}
Si la transition- reconquête de l’industrie française appelle à redimensionner l’intervention publique et à en rebâtir les fondements, elle interpelle  aussi la conception du management industriel à l’œuvre depuis des décennies. Le modèle libéral, grandement théorisé aux Etats-Unis et au Japon, s’est imposé un peu partout. 

Les groupes, grands et de taille intermédiaire, se sont organisés en « centres de profit », dans l’objectif de  dégager une « valeur pour les actionnaires » égalant ou s’approchant  des rendements que les opérations spéculatives obtiennent sur les marchés financiers.  Au nom de la nécessité pour les entreprises d’être « agiles » sur les marchés, les lignes hiérarchiques ont été raccourcies. Au nom de l’attente des salariés à être mieux reconnus, les objectifs  et l‘évaluation du travail ont été individualisés. Cela s’est accompagné de dispositifs de rémunération individualisés et flexibles indexés sur l’atteinte des objectifs. Ce système managérial a été conçu pour que les objectifs financiers fixés par les conseils d’administration se déclinent en objectifs concrets jusqu’au responsable de l’équipe élémentaire de travail – le manager de proximité –  et ce en un nombre d’étapes réduit afin de limiter les brouillages dans la transmission hiérarchique et de gagner en rapidité d’exécution. 
En outre, la production a été flexibilisée, tant au niveau technique par l’organisation en flux tendus visant la réduction du coût financier des stocks amont et aval\footnote{C’est la conversion de l’industrie au “toyotisme”, organisation productive où l’efficacité est moins recherchée dans la spécialisation des tâches prônée par le taylorisme et le fordisme, mais dans le principe que la production doit s’adapter à la demande en temps réel. L’inventeur de cette méthode, l’ingénieur de Toyota Taiichi Ohno, l’a synthétisée par les fameux cinq zéros : zéro délai, zéro stock, zéro papier, zéro défaut, zéro panne. Le succès du toyotisme résulte de ce que, parti de rien dans les années 1950, Toyota est parvenu à devenir un grand de l’industrie automobile mondiale en une vingtaine d’années. La chasse aux résidus au-dessus des 5 zéros a fait la fortune d’un grand nombre de consultants en management.}, qu’au niveau humain par le recours permanent à un volant d’intérimaires. 

Les politiques du « recentrage sur le cœur de métier », censées éviter l’affaiblissement de la position concurrentielle des entreprises dans leur métier de base,  ont été en réalité imposées par les marchés financiers, qui privilégient les entreprises dites « pure players », c’est à dire concentrées sur un segment de marché unique, car elles sont plus faciles à modéliser et permettent une meilleure lisibilité pour les stratégies spéculatives. Ces politiques se sont en outre concrétisées dans un processus permanent d’externalisation des tâches et activités\footnote{Serge Tchuruk, devenu PDG d’Alcatel-Alsthom en 1995 s’est attaché à recentrer le groupe sur l’activité des télécommunications. Il s’est séparé d’Alsthom et a racheté l’américain Lucent dans l’objectif de faire une entreprise sans usines, c’est-à-dire centrée sur la conception des produits et sous-traitant toute leur production. C’était la version “hard” du recentrage sur le coeur de métier : ne conserver que ce qui garantit le profit. Mais cela a été un échec retentissant. La cote boursière d’Alcatel s’est effondrée et aujourd’hui ce n’est plus qu’une marque sous contrôle chinois. 150.000 emplois ont été supprimés.}, y compris dans les cœurs de métier. Cela a conduit les groupes industriels à constituer des réseaux de sous-traitance, dans lesquels des entreprises sont dominées et contraintes de vendre à prix serré, répercutant les conséquences sociales sur leurs salariés sans que cela engage la responsabilité des donneurs d’ordre. Cela a conduit aussi à justifier que les entreprises filiales sous-traitent des activités de service à leurs sièges sociaux ou à d’autres filiales du groupe, les marges réalisées sur ces services alimentant des transferts de profit s’ajoutant aux versements de « dividendes » des filiales vers le siège ou des sociétés holding, parfois opportunément localisées dans des paradis fiscaux. Les groupes ont délocalisé de nombreuses filiales et entreprises sous-traitantes dans le monde entier, au point qu’une grande part du déficit commercial français résulte des transferts internes aux groupes multinationaux.

Les processus d’externalisation des activités est allé jusqu’à sous-traiter des fonctions stratégiques, comme le contrôle de la qualité des produits. Pour diminuer leurs effectifs de services techniques, beaucoup de grandes entreprises se sont reposées sur des procédures dites « d’assurance qualité » censées établir des relations de confiance entre clients et fournisseurs. Une partie des déboires du chantier de l’EPR de Flamanville résulte de ce qu’EDF a supprimé la direction de l’ingéniérie qui assurait le contrôle des travaux d’équipements en lui subsituant des procédures d’assurance qualité.

Pour concevoir l’industrie des transitions énergétiques et écologiques, il pourrait être utile, voire indispensable, de repenser ces évolutions de l’organisation du travail. Il faudrait notamment démêler ce qui s’est trouvé imbriqué dans ces évolutions, à savoir ce qui permet une réelle amélioration de la performance productive et de la place de l’humain au travail et ce qui relève d’une recherche effrénée du rendement financier de la production et du management par l’amaigrissement des effectifs qui l’accompagne – le « lean management ».  En particulier, le management par objectifs de centres de profit a trop fréquemment été perçu par les employés comme le désintérêt porté par les directions d’entreprise à la qualité de leur travail et à leurs compétences professionnelles et les amène à ne plus trouver de sens positif à leur travail. L’implication de l’industrie dans les transitions énergétique et écologique peut permettre de lui redonner un sens , à condition d’assurer la transition industrielle par un management revu et adapté. 

\subsection{Enjeux sociétaux - à écrire}
\begin{textit}
image de dureté du monde de l’entreprise, faible féminisation de l’emploi.
crainte de l’industrie (pollution, accidents industriels) et rejets loin des lieux de vie
place en territoires
\end{textit}


\section{Vues sur la situation de l'industrie}
\subsection{La désindustrialisation de la France en chiffres}
De 1950 au premier choc pétrolier de 1973, l’emploi industriel augmentait – certes moins que l’emploi total en France. En 1974, l’industrie employait 5 millions de salariés en équivalents temps-plein et après cette date il n’a cessé de régresser. Fin 2016, ce nombre n’est plus que de 2,7 millions, soit une perte de 2,3 millions d’emplois directs. Durant la même période, l’ensemble de l’économie a employé 6 millions de personnes de plus. Résultat : la part de l’emploi industriel dans l’emploi total a régressé de 29\% à  11,7\%.

Certes une partie de cette régression résulte de l’accroissement des activités de services dans l’économie, de l’externalisation de fonctions auparavant assurées par les entreprises industrielles vers des sociétés de service aux entreprises et de gains de productivité liés à l’automatisation et à la robotisation. Mais pour une grande part, elle traduit le recul de la production industrielle en France.

\begin{figure}[H]
\centering
\includegraphics*[width=0.8\textwidth]{images/emploi}
\caption{Source~: INSEE Statistiques – Comptes nationaux annuels – Emploi intérieur}
\label{fig:emploi}
\end{figure}

Le secteur de l’industrie se décompose en « industrie manufacturière » (agroalimentaire ; chimie ; pharmacie ; métallurgie ; matériels de transport ; transformation des matériaux non ferreux ; fabrication de machines et d’équipements ; électronique et optique ; textile, habillement et cuir ; papier, carton et imprimerie…) et 3 autres grandes branches : extraction des ressources du sous-sol (branche devenue marginale en France depuis la fermeture de la plupart des mines) ; production et transport de l’électricité, du gaz et de la vapeur ;  production et distribution de l’eau, assainissement et  traitement des déchets.

Le graphique suivant montre que la régression industrielle provient de l’industrie manufacturière. Cela ne signifie pas que les autres branches sont sans poser de problème. Ainsi la production d’électricité, stratégique pour tout pays, dépend pour ses investissements en moyens de production de la fabrication de biens d’équipements, de la métallurgie et d’autres branches. Celles-ci sont de moins en moins sollicitées avec le moratoire sur le nucléaire et le recours aux énergies éoliennes et solaires, dont les machines sont importées pour l’essentiel.

\begin{figure}[H]
\centering
\includegraphics*[width=0.8\textwidth]{images/valeur-ajoutee}
\caption{Source~: INSEE Statistiques – Comptes nationaux annuels – Comtes de production et d’exploitation par branche}
\label{fig:valeur-ajoutee}
\end{figure}

Cela étant, la part des produits industriels dans la demande intérieure décroît aussi. Il s’agit d’une décroissance relative, résultant de l’augmentation plus rapide de la demande en autres produits: les services immobiliers (formés essentiellement des loyers), les autres services marchands et les services publics.

\begin{figure}[H]
\centering
\includegraphics*[width=0.8\textwidth]{images/demande-interieure}
\label{fig:demande-interieure}
\end{figure}

Le graphique suivant compare les décroissances des parts de l'industrie dans la production intérieure\footnote{La valeur ajoutée de l’ensemble des secteurs est à peu près égale au PIB hors TVA et TIPP.} (exprimée en valeur ajoutée) et des produits industriels dans la demande intérieure\footnote{demande intérieure = consommation des ménages + consommation des administrations + investissements + variation des stocks. La demande « finale » lui ajoute les exportations. Les « emplois » totaux en produits ajoutent les consommations intermédiaires des entreprises à la demande finale.}. Les évolutions sont exprimées en indice (valeur 100 en 1960) afin de pouvoir effectuer une comparaison des évolutions de la demande et de la production.

Jusqu’en 1986, la décroissance de la part de la production industrielle dans la  valeur ajoutée nationale se confondait avec celle de la demande en produits industriels dans la demande intérieure. Mais à partir de 1987, la part de la production industrielle a décliné beaucoup plus. 
Le décrochage de la production est clairement concomitant à la mise en place dès le milieu des années 1980 des politiques  visant la libéralisation de l’économie, la déréglementation financière et le désengagement de l’État en matière de politique industrielle.

\begin{figure}[H]
\centering
\includegraphics*[width=0.8\textwidth]{images/part-industrie}
\caption{INSEE – Comptes nationaux annuels~: Équilibre ressources-emplois et Comptes de production des branches}
\label{fig:part-industrie}
\end{figure}

\subsection{Recours aux importations}
Les besoins nationaux annuels en produits industriels correspondent à ce que la comptabilité nationale nomme les «emplois intérieurs» en ces produits. Ceux-ci sont constitués des consommations et investissements effectués par les agents économiques : entreprises non financières et financières, administrations, ménages\footnote{Les consommations des entreprises sont dites “intermédiaires” (c’est la partie de la production qui s’en déduit pour former la valeur ajoutée). La consommation des ménages et des administrations forme la “consommation finale”. En ajoutant l’investissement et les stocks à la consommation finale, on forme la “demande finale”. En lui ajoutant les consommations intermédiaires des entreprises, on forme les “emplois intérieurs” que nous appellerons ici  « demande intérieure ».\\
Pour chaque produit, la relation d’équilibre des emplois et des ressources s’écrit~:\\
Production + Importations + Marges commerciales, frais de transport et taxes = Demande intérieure + Exportations}. Dans les graphiques qui suivent, les exportations, importations et soldes sont exprimées en valeur relative à la demande intérieure. C’est une manière de visualiser les évolutions hors effets de croissance économique et de hausse des prix.

Le premier graphique visualise d’une part le phénomène de mondialisation de la production industrielle : les échanges avec les pays étrangers, importations et exportations, croissent de façon quasi-continue depuis le début des années 1970 – on constate néanmoins des courtes périodes de régression lors des crises financières et économiques, la plus marquée étant celle de 2008. D’autre part, on y voit comment évolue le solde des échanges. De 1960 à 1982, les importations croissent plus que les exportations et les échanges commerciaux en produits industriels deviennent déficitaires à hauteur de 4\% environ de la demande intérieure en ces produits. De 1982 à 1997 un redressement s’opère, mais la chute reprend ensuite : en 2011, le déficit atteint 5,4\% de la demande intérieure. Un très timide mouvement de reprise semble être intervenu depuis. Mais en 2018  – dernière année où l’INSEE livre les données – le déficit reste au niveau des pires scores de l’histoire, près de 4\% de la demande intérieure.  En valeur monétaire, il est de 58 milliards€, dont 37 milliards au titre des importations de combustible fossile et matières premières\footnote{Le solde importateur de 58 milliards est calculé en intégrant les marges commerciales et de transport aux exportations. Si les exportations sont exprimées comme les importations en prix à la production (“sortie usine”), le déficit est supérieur. Selon l’INSEE, il a été de 91 milliards € en 2016.}.


\begin{figure}[H]
\centering
\includegraphics*[width=0.8\textwidth]{images/importations}
\caption{INSEE – comptes annuels de 2018 – Équilibres emplois-ressources par branche}
\label{fig:importations}
\end{figure}

Les deux dégringolades du solde du commerce extérieur en produits industriels ont toutefois des causes différentes. Cela se voit sur les 2 graphiques suivants. Le premier visualise le solde des combustibles fossiles et matières premières extraites du sous-sol, produits essentiellement importés, le second le solde de l’ensemble des autres produits industriels. On constate sur le premier un creusement important du déficit en combustibles et matières premières durant les années 1960-1970, résultant d’une utilisation de plus en plus importante de produit pétroliers exportés diminuant la part du charbon extrait en France, d’une délocalisation de la production charbonnière et de l’envolée des cours du pétrole après les chocs de 1973 et 1979. Cela explique en grande partie la première dégringolade du solde des produits industriels. Sur le second graphique, on constate que depuis le 21ème siècle, les exportations ont tendance à stagner tandis que les importations continuent de progresser. Ainsi, la seconde dégringolade n’est pas liée à l’alourdissement de la facture énergétique, mais à ce que la demande en produits industriels s’adresse de plus en plus aux importations, faute d’une production nationale suffisante. Cela semble bien la marque des stratégies de délocalisation de nombreuses productions industrielles. Cela sonne comme la faillite du discours politico-économique selon lequel le pays peut abandonner des productions pour se spécialiser dans les secteurs où il a un « avantage comparatif ».


\begin{figure}[H]
\centering
\includegraphics*[width=0.8\textwidth]{images/excedents}
\caption{INSEE – comptes annuels de 2018 – Équilibres emplois-ressources par branche}
\label{fig:excedents}
\end{figure}


\begin{figure}[H]
\centering
\includegraphics*[width=0.8\textwidth]{images/importations}
\caption{INSEE – comptes annuels de 2018 – Équilibres emplois-ressources par branche}
\label{fig:excedents-2}
\end{figure}

Le graphique ci-dessous donne la photographie du solde des échanges par grande classe de produits. L’industrie nationale dégage des excédents dans 6 de ces branches\footnote{Pour l’essentiel, les produits d’une classe sont élaborés par la branche industrielle correspondante. Il y a cependant quelques écarts, les entreprises d’une branche pouvant élaborer des produits d’une classe différente. Ces écarts restent faibles, sauf pour les produits alimentaires qui sont à 10\% élaborés par la branche de l’agriculture (produits élaborés à la ferme). Dans cette note, par simplification, on assimile classe de produit et branche industrielle.}~: les matériels de transport (aéronautique, automobile, ferroviaire, militaire, dans une moindre mesure construction navale)~; les produits agroalimentaires, les produits pharmaceutiques ; la production et la distribution d’électricité et de gaz\footnote{La production de gaz naturel ne relève pas de cette classe, mais des produit extraits du sous-sol.}~; la production et la distribution de l’eau, l’assainissement et les déchets\footnote{Il est possible que le solde exportateur provienne du secteur du traitement des déchets (notamment nucléaires). Ce point reste à vérifier}. Ces 6 branches réunies ont dégagé un solde exportateur de 59 milliards € en 2018.
Dans les 9 autres branches – relevant toutes de l’industrie manufacturière\footnote{L’industrie non manufacturière est constituée de l’extraction des produits du sous-sol et de la production et la distribution d’électricité, de gaz et d’eau, de l’assainissement et du traitement des déchets.} – le solde des échanges est déficitaire. Le déficit cumulé s’est élevé en 2018 à 80 milliards €.


\begin{figure}[H]
\centering
\includegraphics*[width=0.8\textwidth]{images/solde-par-produit}
\caption{INSEE – comptes annuels de 2018 – Équilibres emplois-ressources par branche}
\label{fig:solde-par-produit}
\end{figure}


\subsection{Production nationale insuffisante}
Dans une branche où le solde du commerce extérieur est négatif, intuitivement la production apparaît insuffisante pour couvrir la demande intérieure. Autrement dit, le taux de couverture  de la demande par la production est inférieur à 100\%. En fait, le taux de couverture est inférieur à ce que le solde des échanges commerciaux laisse entendre. La raison résulte de la comptabilisation des marges commerciales, des frais de transport et des taxes dans la comptabilité nationale. La production y est exprimée en « prix sortie usine », les importations en « prix à la frontière »\footnote{dits FAB pour « franco à bord »}. Ce sont des prix hors taxes, frais de transports et marges commerciales des revendeurs. Quant aux exportations, on sait que la TVA ne s’y applique pas, mais elles peuvent être grevées de marges commerciales prises par des revendeurs résidents et de frais de transport à l’intérieur du pays. Le solde du commerce extérieur en produits industriels peut donc être supérieur à sa valeur exprimée en prix à la production.

Pour exprimer la couverture de la demande intérieure par la production intérieure, il convient d’exprimer les 2 termes du ratio de façon cohérente. On peut ainsi choisir d’exprimer la demande hors taxes comme la production et de grever celle-ci des marges commerciales et frais de transport\footnote{C’est un choix de commodité que permettent de réaliser les données fournies par l’INSEE. Il revient à réécrire la relation d’équilibre entre les emplois et les ressources.

Production + Importations + Marges + Taxes – Subventions = Demande intérieure + Exportations

devient~:

[Production – Subventions + Marges sur produits] + [Importations + Marges sur importations] = [Demande intérieure – Taxes] + Exportations

Il faut donc répartir les marges commerciales et de transport sur le territoire national entre les produits élaborés en France et les produits importés. On admet pour cela que les taux de marges sont identiques. Cette hypothèse est évidemment discutable, mais les données de la comptabilité nationale ne permettent pas de faire mieux.} comme l’est la demande.

Le graphique suivant visualise les taux de couverture\footnote{Attention à l’interprétation de ce ratio. Par exemple, un taux de couverture de 100\% ne signifie pas que la demande en un produit est satisfaite seulement par la production nationale. Une partie l’est par des importations, qui sont compensées par les exportations de ce produit. Autrement dit, un taux de couverture de 100\% ne signifie pas une production en autarcie et est compatible avec une ouverture de l’économie.} par classe de produits. Il confirme que pour les branches excédentaires, le taux de couverture est moins flatteur que le solde commercial. La situation des industries pharmaceutiques mérite d’être soulignée : au vu du solde exportateur, on classerait volontiers la branche parmi les « champions » nationaux.  En revanche, la branche apparaît nettement déficiente quant à son taux de couverture de la demande. Ces constats contradictoires en apparence s’expliquent par la conjonction d’un fort degré d’ouverture du secteur – importations et exportations importantes – et d’un niveaux élevé du taux de marge commerciale : 35\%. Seuls les produits textiles (habillement et chaussures) sont revendus à un taux de marge supérieur.

\begin{figure}[H]
    \centering
    \includegraphics*[width=0.8\textwidth]{images/couverture}
    \caption{INSEE – comptes annuels de 2018 – Équilibres emplois-ressources par branche}
    \label{fig:couverture}
\end{figure}

\subsection{Vues sur la situation des branches - pre-covid}
Au vu des éléments précédents, on peut tenter de classer les branches industrielles en plusieurs catégories de performance économique: 

\begin{itemize}
\item les industries « résilientes » qui maintiennent l’emploi, dégagent un solde commercial positif et assurent un taux de couverture de la demande intérieure supérieur à 100\%. Il s’agit de la fabrication des matériels de transport, des industries chimiques, des industries agroalimentaires, de la production et la distribution d’électricité, de gaz et de vapeur et de la distribution de l’eau, de l’assainissement et du traitement des déchets. Dans cette catégorie, certaines industries comme celles du luxe et de  l’armement maintiennent des capacités humaines et des savoir-faire de pointe au niveau mondial, assurent un effet d’entraînement général sur l’outil national de Recherche et le reste de l’industrie, dégagent un solde commercial fortement exportateur  et pourraient être qualifiées de « championnes ». C’était aussi le cas de l’aéronautique jusqu’au tout début de l’année 2020, mais les conséquences de la pandémie du coronavirus fait maintenant planer de fortes incertitudes sur son avenir.
 
\item les industries « en régression » où le solde commercial et le taux de couverture ont reculé depuis le début du 21ème siècle et où l’emploi régresse depuis les années 1980. Il s’agit de  la métallurgie, la fabrication des produits en plastique, caoutchouc, verre et non métalliques, de la branche des bois, papier et imprimerie, d’industries manufacturières diverses comprenant l’installation et réparation de machines et d’équipements 

\item les industries « déficientes » où les solde et taux de couverture ont commencé à reculer depuis les années 1980 et où l’emploi s’est effondré.  Elles sont au nombre de 4 : la fabrication de machines et d’équipements, la fabrication des équipements électriques (dont l’électroménager), la fabrication des matériels informatiques, électroniques et optiques, les industries du textile, de l’habillement et du cuir. 

\begin{figure}[H]
    \centering
    \includegraphics*[width=0.8\textwidth]{images/couverture2}
    \label{fig:couverture2}
\end{figure}


\begin{figure}[H]
    \centering
    \includegraphics*[width=0.8\textwidth]{images/couverture3}
    \label{fig:couverture3}
\end{figure}

\item l’industrie pharmaceutique pourrait compter parmi les « résilientes » au vu du solde commercial qu’elle dégage. Mais sa capacité à couvrir la demande intérieure la situerait plutôt comme une branche en régression, ainsi que l’évolution récente du niveau d’emploi qu’elle assure. Son efficacité du point de vue de l’intérêt national a été fortement interrogée par les pénuries de médicament intervenues dès 2019 et son incapacité à fournir les tests qui auraient été nécessaires pour combattre la pandémie du Covid 19. Elle apparaît comme un secteur stratégique pour la nation et mérite que l’on s’y intéresse en tant que telle.*les industries de l’extraction des combustibles fossiles et matières premières ainsi que celles du raffinage et de la cokéfaction des produits pétroliers méritent aussi d’être analysées en tant que telles. Ce sont en effet des secteurs très fortement importateurs  dont on ne peut imaginer le redressement de la couverture de la demande par une production nationale, étant donnés la pauvreté des gisements pétroliers et gaziers en France et les objectifs de réduire drastiquement la consommation de ces produits dans le cadre de la lutte contre le changement climatique.


\begin{figure}[H]
    \centering
    \includegraphics*[width=0.8\textwidth]{images/couverture4}
    \caption{INSEE – comptes annuels de 2018 – Équilibres emplois-ressources par branche}
    \label{fig:couverture4}
\end{figure}

\begin{figure}[H]
    \centering
    \includegraphics*[width=0.8\textwidth]{images/couverture5}
    \caption{INSEE – comptes annuels de 2018 – Équilibres emplois-ressources par branche}
    \label{fig:couverture5}
\end{figure}

\item les industries de l’extraction des combustibles fossiles et matières premières ainsi que celles du raffinage et de la cokéfaction des produits pétroliers méritent aussi d’être analysées en tant que telles. Ce sont en effet des secteurs très fortement importateurs  dont on ne peut imaginer le redressement de la couverture de la demande par une production nationale, étant donnés la pauvreté des gisements pétroliers et gaziers en France et les objectifs de réduire drastiquement la consommation de ces produits dans le cadre de la lutte contre le changement climatique.

\end{itemize}

\subsubsection{Des éléments pour approfondir}

Pour les organisations patronales et les économistes libéraux, les difficultés de l’industrie française résulteraient de la lourdeur des charges grevant les marges des entreprises. Le coût du travail serait trop élevé. Or globalement dans l’industrie, le taux de marge\footnote{mesuré comme l’excédent brut d’exploitation (valeur ajoutée brute diminuée de la masse salariale et des impôts à la production) rapporté à la valeur ajoutée brute.} atteint depuis 2016 le niveau de 39\%, identique à celui de la période 1998-2004 et supérieur à celui de l’ensemble des sociétés non financières (31\% en 2018). Il était certes descendu à 33\% après la crise des « subprimes », mais ce niveau restait plus élevé que celui de la période 1978-1983.  Certes dans les branches résilientes il fluctue autour de 45\% et est supérieur à celui des autres branches, mais il est plus élevé dans les branches déficientes que dans celles en régression. Il se situe à un niveau très élevé dans l’industrie pharmaceutique –- plus de 60\% -- ce qui n’empêche pas la couverture de la demande intérieure en produits pharmaceutiques de régresser. Ces constats apportent un démenti au fait que les difficultés de l’industrie seraient dues au niveau trop élevé du coût du travail.

\begin{figure}[H]
    \centering
    \includegraphics*[width=0.8\textwidth]{images/marge}
    \label{fig:marge}
\end{figure}

Les marges servent essentiellement à financer les investissements, rémunérer le capital – dividendes versés aux actionnaires et remboursement des emprunts – et payer les impôts sur le bénéfices. En général, elles sont insuffisantes pour financer ces trois postes. Les entreprises sont donc confrontées à un « besoin de financement », couvert soit par de nouveaux emprunts, soit par des apports des actionnaires en capital. On peut évaluer « l’effort d’investissement » d’une entreprise en rapportant son montant à la marge d’exploitation. On constate que dans l’industrie cet effort est à peu près stable de 1980 jusqu’à 2003-2004 – entre 60 et 70\% –  et qu’il se redresse depuis à presque 80\%. C’est dans la branche des industries pharmaceutiques qu’il est le plus faible – il est vrai que les marges y sont très élevées –- et c’est paradoxalement dans les branches déficientes qu’il est le plus élevé. Cela s’explique, comme on va le voir, par le caractère capitalistique de cette branche et la forte consommation de capital de ses activités productives.

\begin{figure}[H]
    \centering
    \includegraphics*[width=0.8\textwidth]{images/investissement}
    \label{fig:investissement2}
\end{figure}

En première analyse, l’investissement peut être décomposé en deux parts. La première équivaut au remplacement à l’identique du capital fixe consommé, c’est-à-dire des machines, équipements et bâtiments arrivés en fin de vie~; la seconde équivaut au capital fixe supplémentaire afin d’assurer une croissance productive – c’est l’investissement net. Dans la réalité le capital fixe se reproduit rarement à l’identique~: les produits et les techniques de production évoluent et les équipements productifs nouveaux doivent s’adapter à ces évolutions. Ainsi l’investissement  net  peut être destiné aussi bien à la croissance de la production qu’à celle de la productivité.

Les 2 graphiques suivants présentent les évolutions de la consommation de capital fixe et de l’investissement net dans l’industrie, par catégorie de branche. Ces deux paramètres son exprimés en pourcentage du capital fixe brut\footnote{Le capital  fixe brut exprime le cumul des investissements toujours en service, en valeur actualisée. Le capital fixe net est obtenu en lui déduisant le montant du capital fixe consommé dans le processus de production, qu’on pourrait appeler l’amortissement économique –- à ne pas confondre avec l’amortissement comptable, qui est une notion fiscale.} immobilisé. 

\begin{figure}[H]
    \centering
    \includegraphics*[width=0.8\textwidth]{images/investissement}
    \label{fig:investissement}
\end{figure}

\begin{figure}[H]
    \centering
    \includegraphics*[width=0.8\textwidth]{images/investissement-net}
    \label{fig:couverture}
\end{figure}

On y observe une tendance générale dans l’industrie à l’augmentation du taux de consommation du capital fixe, ce qui traduit une tendance à la diminution de la durée de vie ou d’utilisation des équipements.   Les branches résilientes sont celles où la durée de vie des équipements est la plus longue, les branches déficientes celles où elle est la plus courte. Cela pourrait constituer une première explication de l’effort plus élevé d’investissement dans les secondes que dans les premières.

Quant à l’investissement net, le classement des branches est inversé et rejoint l’intuition : c’est dans les branches résilientes qu’il est le plus élevé et dans les déficientes le plus faible, les branches en régression se trouvant dans une situation intermédiaire. D’une manière générale, il apparaît que de 1990 à 2010 l’investissement net a reculé dans l’industrie, avec une tendance à la reprise depuis : stabilité dans les branches résilientes, faible reprise dans les branches en régression, forte reprise dans les déficientes. Mais ces deux dernières ont subi l’épreuve de périodes d’investissement net négatif : de 2009 à 2014 pour les branches en régression, de 2002 à 2014 pour les déficientes, ce qui signifie une dégradation des outils de production dans ces branches. Cette dégradation a fortement touché les branches déficientes. Dans l’industrie pharmaceutique, l’investissement net a été le plus élevé  jusqu’à la crise des subprimes et s’est effondré depuis, jusqu’à devenir négatif après 2013, bien que le taux de marge ait poursuivi sa course en tête au-dessus de 60\%. On en déduit que ces dernières années les actionnaires de l’industrie pharmaceutique se servent encore plus copieusement qu’auparavant et/ou utilisent les excédents d’exploitation pour investir encore plus à l’étranger au détriment de la pérennité de l’outil de production sur le sol national.

Le graphique suivant montre les évolutions de l’efficacité du capital dans les branches industrielles. C’est le ratio de valeur ajoutée au capital fixe brut\footnote{Les économistes libéraux appellent ce ratio productivité du capital par parallélisme avec la productivité du travail – rapport de la valeur ajoutée à la masse salariale. Ils considèrent en effet que le capital est un facteur de production au même titre que le travail. Mais, si on pense que seul le travail produit de la valeur et que le capital est un outil de production sans être directement productif, le terme efficacité convient mieux à qualifier le rapport de la valeur ajoutée au capital.}. Il diminue dans toutes les catégories de branches industrielles : dans l’ensemble de l’industrie, il est passé de plus de 40\% à moins de 30\% de 1978 à 2018. Dans les industries résilientes et en régression,  il a chuté d’un tiers depuis le début du 21ème siècle : de 30\% à 20\% pour les premières, de 60\% à 40\% pour les secondes. Dans les branches déficientes, la chute a démarré 20 ans plus tôt et est encore plus vertigineuse, l’efficacité du capital étant passée de 65\% en 1978 à 27\% en 2018.


\begin{figure}[H]
    \centering
    \includegraphics*[width=0.8\textwidth]{images/efficacite-capital}
    \label{fig:efficacite-capital}
\end{figure}

Pour maintenir une valeur ajoutée identique, la baisse de l’efficacité du capital devrait être compensée par un investissement net positif. Concernant les branches en régression, l’efficacité du capital décroissant d’environ 1\% par an depuis 1978, le taux d’investissement net devrait être supérieur à 1\%, ce qui n’est pratiquement plus arrivé depuis 1992. Ainsi, si l’effort d’investissement dans ces branches apparaît important, c’est un faux-semblant. 

Ce n’est pas seulement le niveau d’investissement qui est interrogé, mais aussi sa nature. En effet,  la baisse de l’efficacité du capital peut résulter de stratégies industrielles délibérées, visant plus à accroître la productivité du travail et la réduction des coûts des produits que le volume de la production. Redresser l’industrie française passerait alors par la réorientation du contenu des investissements, visant plus de valeur ajoutée, plus d’emplois et une moindre réduction de l’efficacité du capital. Cette interrogation concerne au premier chef les branches en régression, mais vaut aussi pour les autres.  

\textit{Point à approfondir : dans quelle mesure et dans quels cas la baisse de l’efficacité du capital résulte-t-elle de stratégies de réduction des coûts de fabrication des produits pour résister à la concurrence internationale ?}

\subsubsection{Remarque annexe}
Les tendances simultanées observées dans l’industrie à une relative stabilité des taux de marge et une baisse de l’efficacité du capital aboutissent logiquement à la diminution du taux de profit issu de la production, celui-ci étant le rapport de la marge d’exploitation au capital fixe immobilisé. Cette diminution est particulièrement nette dans les branches en régression depuis le milieu de la décennie 1980 et s’est généralisée à l’ensemble de l’industrie depuis le début du 21ème siècle. Cela peut expliquer pourquoi les grands groupes se « financiarisent » en se tournant vers les marchés financiers pour trouver des rentabilités plus élevées dans la spéculation que dans la production. 

\begin{figure}[H]
    \centering
    \includegraphics*[width=0.8\textwidth]{images/profit}
    \label{fig:profit}
\end{figure}

\subsection{Analyse détaillée des branches et pistes de redressement}
\subsubsection{Les industries « résilientes »}

\begin{figure}[H]
    \centering
    \includegraphics*[width=0.8\textwidth]{images/solde2}
    \label{fig:solde2}
\end{figure}


\begin{figure}[H]
    \centering
    \includegraphics*[width=0.8\textwidth]{images/etp2}
    \label{fig:etp2}
\end{figure}

\paragraph{Fabrication des matériels de transports\protect\footnote{aéronautique, spatial, ferroviaire, automobile, construction navale, construction de matériels roulants de combat}}

Prises dans son ensemble, cette branche est exportatrice, rapportant un excédent annuel de l’ordre de 50 milliards €. Elle rassemble des filières où l’excellence de la France est reconnue dans le monde, tout particulièrement dans les domaines de l’espace avec Ariane et de l’aéronautique avec Airbus – groupe franco-germano-hispano-britannique –, des avions militaires avec Dassault.  L’industrie ferroviaire compte aussi parmi les atouts historiques de l’industrie française. Avec le lancement en février 2020 de l’opération d’achat de Bombardier-Transport, Alstom devrait devenir le 2ème constructeur ferroviaire mondial après le géant chinois CRRC\footnote{La composition de l’actionnariat du nouveau groupe constitue néanmoins un point de vigilance. Le premier actionnaire, avec 18\% des parts, est l’investisseur canadien – québécois – CDPQ. Bouygues arrive en second avec 10\% du capital. Pour financer l’achat de Bombardier-Transport, Alstom doit lever 2 milliards d'euros sur les marchés. Le contrôle du groupe reste peu clair. La direction d’Alstom promet une “synergie de coûts” de 400 millions d'euros par an dans 5 ans et un rendement des actions à 2 chiffres.}.  

\paragraph{Automobile}
L’industrie automobile offre un visage très différent. Elle pouvait encore être considérée «résiliente»  en 2005, mais depuis les stratégies de délocalisation l’ont déclassée de 2 niveaux. Avec un solde des échanges importateur de 9 milliards € en 2017 et un taux de couverture passé sous la barre des 80\%, l’industrie automobile française compte désormais parmi les branches en difficulté, et ce avant même la survenue de la crise consécutive  la pandémie du coronavirus. Si elle l’est du point de vue de la nation, ce n’est sans doute pas l’avis des actionnaires qui ont bénéficié de confortables versements de dividendes. En tant que multinationales, Renault et PSA se portent bien et voient leurs ventes s’accroître dans le monde comme en France. Elles ont été créatrices d’emplois au niveau mondial, mais ont détruit des dizaines de milliers d’emplois dans leur pays\footnote{Voir le blog de Gabriel Colletis dans Mariane : https://www.marianne.net/debattons/tribunes/ces-groupes-automobiles-francais-qui-se-moquent-eperdument-du-made-france}. Aujourd’hui, acheter une automobile de marque française ne signifie plus acheter français.

Les stratégies d’évolution doivent veiller à maintenir un équilibre entre voitures électriques et hybrides d’une part, et motorisations et véhicules thermiques d’autre part, qui constituent des points forts traditionnels de l’industrie française. Les premières annonces faites par le gouvernement peuvent en ce sens paraître irréfléchies, et faire une trop grande part à la « coopération » franco-allemande. 

\begin{figure}[H]
    \centering
    \includegraphics*[width=0.8\textwidth]{images/auto}
    \label{fig:auto}
\end{figure}

\paragraph{Construction navale}
Au niveau mondial, la construction navale est concentrée dans 3 pays asiatiques: la Chine, la Corée du Sud et le Japon, qui produisent 90\% des capacités de transport maritime. Avec les Chantiers de l’Atlantique, la France dispose cependant du plus grand site de construction navale d’Europe, spécialisé dans la construction des paquebots (navires de croisières et ferries) et des navires militaires, les navires de marchandises (navires de vrac, porte-conteneurs, pétroliers, méthaniers..) n’étant pas fabriqués en Europe. Le développement des éoliennes marines a permis aux Chantiers de l’Atlantique d’investir le champ nouveau de la construction des sous-stations immergées centralisant l’évacuation de l’énergie électrique des parcs éoliens. L’entreprise a été nationalisée en 2018 après le désengagement du Coréen STX qui en contrôlait le capital et l’échec des négociations avec le groupe public italien Fincantieri . Mais le désengagement de l’État et la cession des chantiers de l’Atlantique à Fincantieri sont en cours d’examen par la Commission européenne.

La crise économique consécutive à la pandémie du Covid 19, aux mesures de confinement prises au niveau mondial et à l’effondrement de la demande qui en a résulté, modifie sérieusement la donne. Une grande incertitude pèse sur ce que deviendront les flux de déplacements de personnes au niveau mondial. Finiront-ils par se rétablir au niveau antérieur à la pandémie ? Peut-on continuer à envisager leur croissance ou bien les craintes sanitaires limitant les voyages seront-elles durables ?  Quant aux flux de marchandises, les tendances à la relocalisation de productions, souhaitables pour des raisons tant économiques qu’écologiques, porteront-elles un coup d’arrêt à leur croissance jusqu’à présent ininterrompue ? Le simple fait que ces questions se posent suggère que la France ne peut pas  centrer sa politique de redressement industriel sur l’accroissement continu des exportations d‘avions et de navires de croisière, deux des domaines où elle dispose d’un avantage compétitif dans le monde.

En revanche, les transitions énergétique et écologique appellent à repenser les moyens de la mobilité. On peut identifier les grandes cibles des transitions à opérer~:
\begin{itemize}
\item le développement du transport par rail des personnes comme des marchandises, alors que les politiques publiques et stratégies économiques privées ont jusqu’à présent privilégié le transport routier~;
\item le développement des véhicules individuels électriques (voitures et deux roues) plutôt adaptés aux déplacements de courte distance et des véhicules hybrides rechargeables, à très faible consommation d’hydrocarbures, de relevant de technologies pouvant devenir matures dans le futur (hydrogène, pile à combustible…)~;
\item le développement du transport fluvial pour le fret~;
\item une nouvelle conception de la propulsion des navires de vrac et des porte-containers afin qu’ils deviennent faiblement émetteurs de gaz carbonique. La France, qui dispose d’une large façade maritime, doit se doter d’une industrie navale conséquente. Elle pourrait mettre à profit l’un de ses atouts : son savoir-faire en matière de propulsion nucléaire des navires militaires qui pourrait être transposé dans le domaine civil afin de répondre à la demande nouvelle de décarboner le transport maritime.

\end{itemize}

L’avenir des industries des matériels de transport est fortement lié aux développements des infrastructures adaptées aux nouvelles mobilités (réseau de recharge des batteries des véhicules électriques, voies ferrées et infrastructures ferroviaires intégrant le ferroutage, voire le transport des véhicules personnels\footnote{Le transport des voyageurs accompagnés de leurs véhicules automobiles existait mais a été abandonné. Pourtant, il peut s’avèrer un auxiliaire nécessaire à l’usage du véhicule électrique, dont le développement est à l’ordre du jour mais qui reste peu adapté aux grands déplacements – du fait notamment du temps de recharge des batteries et de l’importance de la capacité des stations de recharge à installer le long des routes pour répondre aux besoins de pointe. La France dispose de grandes compétences et d’un outil industriel performant en matière de fabrication de matériels ferroviaires et de constructions de voies.}, développement d’infrastructures de production et de distribution d’énergie électrique répondant à un usage croissant de l’électricité pour la mobilité –- ainsi que d’autres usages).

\paragraph{Industries agroalimentaires}
\textit{à écrire}

\paragraph{Industries chimiques}
\textit{à écrire}

\paragraph{Production et distribution d’électricité, de gaz et de vapeur}
\textit{à écrire}

\paragraph{Production et distribution d’eau, assainissement, traitement des déchets, dépollution}
\textit{à écrire}

\subsubsection{Les industries en régression}

\begin{figure}[H]
    \centering
    \includegraphics*[width=0.8\textwidth]{images/solde3}
    \label{fig:solde3}
\end{figure}

\begin{figure}[H]
    \centering
    \includegraphics*[width=0.8\textwidth]{images/etp3}
    \label{fig:etp3}
\end{figure}

\paragraph{Métallurgie}
\textit{à écrire}

\paragraph{Fabrication des produits en plastique, caoutchouc, verre et matières non métalliques}
La métallurgie et la fabrication de produits non métalliques (caoutchouc, plastiques, verre, céramiques et autres produits) comptent des atouts. Pour autant, la production ces branches ne couvre que 87\% et 80\% respectivement de la demande intérieure des ménages, entreprises et administrations pour ces produits. Le solde des échanges extérieurs commerciaux de ces 2 branches est de -14 milliards d'euros, alors qu’il était excédentaire pour chacune d’elle des années 1950 jusqu’à la fin du 20ème siècle.

\paragraph{Industries du bois (hors fabrication de meubles) , du papier, imprimeries}
\textit{à écrire}

\paragraph{Installation et réparations de machines et équipements}
\textit{à écrire}

\paragraph{Diverses activités manufacturières\protect\footnote{fabrication de meubles, d’instruments et fournitures médicales, joaillerie, articles de sports, jouets ...}}

\subsubsection{Les branches déficientes}

\begin{figure}[H]
    \centering
    \includegraphics*[width=0.8\textwidth]{images/solde4}
    \label{fig:solde4}
\end{figure}

\begin{figure}[H]
    \centering
    \includegraphics*[width=0.8\textwidth]{images/etp4}
    \label{fig:etp4}
\end{figure}

\paragraph{Fabrication de machines et équipements}
La fabrication des biens d’équipements, les activités d’installation et de réparation des machines font partie des faiblesses de l’industrie française. La production nationale couvre moins de 60\% de la demande intérieure et le déficit commercial est de près de 7 milliards d'euros. S’y ajoute un déficit du même ordre pour l’installation et la réparation des machines et installations. 

\paragraph{Les équipements du système de production d’électricité}
Ainsi au 20ème siècle la France fabriquait les moyens de sa production d’électricité (centrales à charbon, à fuel et à gaz, hydraulique, nucléaire). Aujourd’hui, les éoliennes et panneaux solaires sont pour l’essentiel importés. Pourtant, des tentatives d’édification d’industries nationales en matière de nouvelles énergies renouvelables ont existé. Areva avait investi le domaine de la production éolienne à travers sa filiale Areva-Wind. Faute de commande et de soutien public et dans un contexte de lourdes difficultés financières de la maison-mère, ces activités ont été vendues en 2015 pour une bouchée de pain à Siemens, devenu 4 ans après le fournisseur principal des parcs éoliens marins français.  Alstom a développé la machine Héliade également destiné à l’éolien en mer. Mais Alstom-Wind a été vendu avec le reste de sa branche énergie à General Electric. Aujourd’hui une grande partie d’Heliade n’est pas fabriquée en France. Total s’était lancée dans la fabrication de panneaux solaires, avec Sunpower, et EDF a pris le contrôle de Photowatt,  mais ces entreprises n’ont pas pas résisté au dumping chinois, dans le cadre d’un marché européen ouvert où la concurrence est libre – mais en l’occurrence complétement faussée. Résultat : les près de 8 milliards annuels de subventions publiques soutenant le développement des énergies éoliennes et solaires profite aujourd’hui essentiellement à des entreprises étrangères\footnote{Pour plus de détails, se reporter à l’excellente étude de la lettre géopolitique de l’électricité n°101 de mai 2020, publiée par Lionel Taccoen (http://www.geopolitique-electricite.com/)}. 

L’industrie nucléaire, qui avait acquis une réputation mondiale après avoir été développée grâce à la recherche menée au CEA, l’ingénierie d’EDF et la constitution de fleurons industriels comme Framatome et Alstom, est aujourd’hui en grande difficulté. Les déboires de l’EPR de Flamanville en témoignent, beaucoup de compétences ont été perdues. Cela résulte de la conjonction de décisions désastreuses : la volonté mitterrandienne d’européaniser l’industrie nucléaire par l’abandon de la filière française dite « N4 » et la mise en chantier d’un nouveau prototype associant Areva et Siemens, qui a finalement abandonné le projet avec le renoncement de l’Allemagne au nucléaire ; l’abandon par EDF de ses services d’ingénierie et de contrôle des chantiers, qui a conduit à ce que des défauts majeurs de l’EPR ne soient révélés qu’au stade du contrôle par l’autorité de sûreté nucléaire ; des politiques publiques renonçant volontairement à donner une quelconque perspective au renouvellement du parc électronucléaire d’EDF. Les compétences acquises depuis les années 1950 dans la filière  à neutrons rapides, une première fois mises à mal par la fermeture prématurée de la centrale  Superphénix sous le gouvernement Jospin, viennent de subir un nouveau coup avec le renoncement d’Emmanuel Macron à lancer le projet Astrid, ce qui a conduit le CEA à démanteler l’essentiel des équipes de recherche sur la filière française de génération 4. La Russie se trouve ainsi en position forte sur le nucléaire de demain et les Etats-Unis viennent de décider en 2020 d’investir plus avant ce terrain. 

\paragraph{Les équipements de l’industrie : machines-outils, robotique et productique}
La machine-outil est le symbole de la machine industrielle car elle intervient dans tous les process de fabrication des équipements manufacturés. Elle réalise l’usinage de la matière : perçage, taraudage, rectification, etc. Les pièces produites une fois assemblées formeront les équipements (pompes, moteurs…) qui à leur tour formeront les produits finaux du quotidien (voitures, avions, lave-linge…). A l’origine machine conventionnelle commandée par un humain, la machine-outil est devenue grâce à l’intégration de la commande numérique un centre d’usinage multi opération, avant de devenir la machine intelligente de l'usine du futur interconnectée. La production de biens manufacturés dépend donc directement de celle des machines-outils, qui est devenue une activité stratégique.

Or en 2018 l’ensemble des entreprises regroupées par le SYMOP\footnote{SYMOP: Syndicat des machines-outils et de la productique. Membre de la Fédération des Industries Mécaniques (FIM) et fondateur de l’Alliance Industrie du Futur (AIF), le SYMOP est l'organisation professionnelle des créateurs de solutions industrielles, fabricants de machines, technologies et équipements pour la production industrielle} dans 9 catégories de machines affiche un solde négatif entre exportations et importations de 862 millions d’euros. La part la plus importante de l’investissement 2018 en France de 1 204 millions d’euros appartient à la catégorie « machines-outils métal » qui produit en France 739 millions d’euros pour exporter 560 millions et importer 1025 millions soit un solde négatif de 465 millions d’euros. Suivent les machines et systèmes d’assemblage, seule catégorie avec un solde positif de la balance commercial de +163 millions d’euros. Viennent ensuite les machines d’emballage et de conditionnement, solde négatif -111 millions d’euros. 

De même, l’industrie mondiale se robotise. Le Japon se maintient en 2019 comme le leader mondial avec 4 entreprises qui fournissent 52\% des robots installés. Le chiffre d'affaires mondial des ventes de robots avoisine les 16 milliards d’euros en 2017.  Les 10 pays les plus automatisés au monde sont la Corée du Sud, Singapour, l'Allemagne, le Japon, la Suède, le Danemark, les États-Unis, l'Italie, la Belgique et Taïwan, la France n’apparaissant qu’au 18éme rang. On compte en Europe des entreprises telles que ABB, COMAU, KUKA, UNIVERSAL  ROBOTS, STAUBLI, mais l’industrie française est pauvre en fabrication de robots. La moyenne mondiale de robots installés en 2017 et de 74 unités pour 10 000 salariés (309 pour l'Allemagne, 132 pour la France et 631 pour la Corée du Sud)\footnote{Source : World Robot Statistics 2017, publiées par la Fédération Internationale de Robotique (IFR)}.

Cette situation est à mettre en relation directe avec les périodes d’investissement net négatif observées de 2009 à 2014 pour les branches en régression et de 2002 à 2014 pour les branches déficientes (cf sopra), qui se sont traduites par une dégradation marquée de l’outil de production.

\subparagraph{Conclusions}
Les biens d’équipement sont le substrat de l’industrie manufacturière puisqu’ils en composent les outils de production. Le redéploiement industriel doit aller de pair avec celui des machines associées. Ainsi, à bien des égards, cette branche peut être qualifiée de stratégique et doit bénéficier, plus que d’un soutien, d’une véritable impulsion politique et économique.

\emph{Il faut cependant noter que renforcer une catégorie de machines industrielles ne convient que si cette dernière s'intègre dans une filière industrielle porteuse d’avenir. Il s’agira donc en priorité d’identifier les filières du futur pour en tirer les typologies de machines associées. Par exemple a priori le développement de l'imprimante 3d sera prioritaire sur celui de la fraiseuse conventionnelle.}

\paragraph{Fabrication d’équipements électriques}
L’industrie de la fabrication des équipements électriques apparaît sinistrée.  Dans cette classe de produits, les échanges avec l’étranger étaient équilibrés jusqu’en 2008, mais depuis un déficit s’est creusé pour atteindre aujourd’hui 7 milliard € en solde et la production ne couvre que 60\% de la demande intérieure. La production d’équipements électriques sobres et durables étant au cœur des exigences des transitions énergétique et écologique, ce pourrait être un motif de redynamiser cette branche sur des objectifs de sobriété et de durabilité.

\paragraph{Fabrications d’équipements informatiques, électroniques, optiques}
« Révolution informationnelle », « transition numérique », les mots ne manquent pas pour qualifier les mutations induites par les progrès technologiques de l’informatique et des réseaux. Ces produits occupent aujourd’hui une place décisive dans l’économie et dans la vie. Or en la matière, l’industrie française n’est pas à la hauteur des compétences scientifiques que le pays a su développer depuis longtemps. Dans les branches de la fabrication de produits informatiques, électroniques et optiques, le déficit s’élève à 15 milliards € par an et seuls 70\% de la demande intérieure est couverte par la production. 

La France a littéralement capitulé dans le secteur stratégique des télécoms. Elle a froidement bradé son champion national ALCATEL, en se payant de mots sur une soi-disant alliance avec NOKIA dont il fut vite clair qu’il ne s’agissait que d’un rachat et d’une liquidation à terme, obérant le futur de l’écosystème. La situation est devenue paradoxale, la France disposant d’une infrastructure de Recherche et d’enseignement supérieur au meilleur niveau international, d’un réseau de PME et start-ups innovantes, d’une politique de soutien public à l’innovation généreux, mais ne disposant plus d’industriel de taille significative susceptible d’offrir des débouchés pérennes à ces investissements. Le maintien du statu quo conduit à offrir sans contrepartie aux industriels étrangers une force de travail formée et des résultats de recherche de qualité. L’alternative possible est de s’appuyer sur ces acquis, tant qu’ils existent, pour relancer le secteur de façon volontariste. 

\paragraph{Industries du textile, de l’habillement et du cuir}
La France s’était habituée à dépendre d’importations pour satisfaire une grande partie de ses besoins en textile, habillement et cuir.  Les délocalisations de production en la matière ont commencé il y a 40 à 50 ans. Mais les défauts de masques et de blouses nécessaires face à la pandémie du coronavirus viennent de montrer le besoin essentiel d’industries textiles localisées. 

Concernant les produits textiles et l’habillement, la conséquence des délocalisations dans les pays à faible coût de main-d’œuvre conduisent à un déficit commercial important (plus de 13 milliards €) et un faible taux de couverture (50\%).  Néanmoins, le taux de marge commerciale appliqué à ces produits est le plus élevé de toutes les branches. 

\begin{figure}[H]
    \centering
    \includegraphics*[width=0.8\textwidth]{images/marge-textile}
    \label{fig:marge-textile}
\end{figure}

\subsubsection{Industrie pharmaceutique}

\begin{figure}[H]
    \centering
    \includegraphics*[width=0.8\textwidth]{images/solde5}
    \label{fig:solde5}
\end{figure}

\begin{figure}[H]
    \centering
    \includegraphics*[width=0.8\textwidth]{images/etp5}
    \label{fig:etp5}
\end{figure}

L’industrie pharmaceutique présente la particularité de ne couvrir la demande intérieure qu’à 88\%, alors qu’elle est excédentaire de 5 milliards d'euros en solde commercial avec l’étranger.  Cela résulte de taux de marges commerciales très élevés appliquées tant aux ventes en France qu’aux exportations : elles sont comprises entre 50 et 60\%! L’industrie pharmaceutique nationale s’est trouvée sous les feux de l’actualité dès 2019 avec l’apparition de pénurie de certains médicaments, notamment en raison de la délocalisation en Inde et en Chine de la production de principes de base. On peut ainsi supposer que les groupes pharmaceutiques ont concentré leur activité sur les productions à forte marge commerciale, en délocalisant des productions moins lucratives, pourtant essentielles dans l’élaboration des médicaments. A l’automne 2020, l’industrie pharmaceutique s’est révélée incapable de produire les tests du COVID 19 en nombre suffisant pour permettre une autre stratégie sanitaire que 2 mois de confinement. 

Des alertes se sont apparues quant au risque de rupture d’approvisionnement de médicaments vitaux comme les anesthésiants nécessaires en réanimation. Un drame a été évité de justesse, soulignant ici encore la folie ayant consisté à dépendre massivement de la Chine et d’une poignée de pays pour la production des principes de base. Dans ces conditions, la question du contrôle national sur l’industrie pharmaceutique a été reposée.
Il faut également noter que le choix politique de payer toujours moins cher les médicaments tombés dans le domaine public est pour une large part à la racine des délocalisations dont nous constatons aujourd’hui les conséquences désastreuses. Ces choix avaient accoutumé le pays depuis des années à supporter des pénuries de certains médicaments, ce qui était autrefois une caractéristique de pays sous développés. 

\subsubsection{Combustibles fossiles, minerais, raffinage et cokéfaction}

\begin{figure}[H]
    \centering
    \includegraphics*[width=0.8\textwidth]{images/solde6}
    \label{fig:couverture}
\end{figure}

\begin{figure}[H]
    \centering
    \includegraphics*[width=0.8\textwidth]{images/etp6}
    \label{fig:couverture}
\end{figure}

Depuis les fermetures des mines de charbon et l’épuisement des gisements de gaz naturel et du fait de l’absence de gisements pétroliers dans le sous-sol français, les combustibles fossiles sont importés en quasi-totalité. 

Les activités de raffinage et de cokéfaction comptent parmi les grandes branches françaises déficitaires (10 milliards € de déficit commercial et 75\% de couverture de la demande). On ne peut évidemment pas viser un rééquilibrage des échanges.  Les pays exportateurs de pétrole aspirent légitimement de plus en plus à exercer les activités de raffinage plutôt qu’exporter le brut. 

Le plus grand des grands enjeux de la transition énergétique est de bâtir une économie et une société qui se passent des combustibles fossiles. Certes, les besoins en produits pétroliers ne vont pas disparaître du jour au lendemain, mais la reconversion des sites de raffinage et cokéfaction mérite d’être pensée par anticipation et préparée. L’émergence de combustibles de synthèse  se substituant au fuel (hydrogène) pourrait être une piste.

Il reste que la France – comme d’autres pays européens – a aussi abandonné l’extraction de son sous-sol de divers minerais (uranium, terres rares, ardoise…), essentiellement parce que leur extraction est moins onéreuse à l’étranger. Or les transitions énergétiques et numériques sont gourmandes en matières premières, matériaux classiques comme l’acier et le béton qui nécessitent l’extraction de fer, de calcaire, d’argile et de sable, mais aussi métaux nobles (cuivre, nickel…) et terres rares. Cela provoque l’épuisement mondial des gisements les plus riches en minerai et oblige à exploiter des gisements à teneur décroissante. Les quantités d’énergie dépensés par les activités d’extraction et  de raffinage augmentent et par voie de conséquence les coûts. Les exigences écologiques grandissent partout dans le monde, ce qui éthiquement n’autorise plus les stratégies visant à délocaliser les pollutions occasionnées par les activités minières. Il convient de se préparer dès aujourd’hui à un avenir où le surcout de l’extraction propre sera intégré dans le coût normal des minerais. Certes, la montée en puissance des activités de recyclage devra et pourra limiter l’extraction des matières premières, mais ne pourra s’y substituer qu’en partie. La politique de délocalisation systématique des activités minières se trouve donc sérieusement interrogée, en France comme en Europe.  

\section{Les conditions du redressement et de la transition}
\subsection{Évaluation de la production nécessaire}
Le déficit commercial de l’économie nationale en produits industriels est de 96 milliards € si l’on comptabilise les échanges en prix à la production et de 58 milliards € lorsqu’on intègre les marges commerciales\footnote{La valeur du solde en prix à la production est issue du tableau des entrées et sorties symétriques de l’INSEE (le plus récent concerne 2016). La valeur du solde marges commerciales, de transport, impôts, taxes et subventions comprises est issu du tableau des équilibres entre les emplois et les ressources en produits (le plus récent concerne 2018).}. Pour le combler “toutes choses égales par ailleurs”\footnote{C’est-à-dire en supposant la demande intérieure et les exportations inchangées. La relation d’équilibre entre les emplois et  les ressources de produits s’écrit~:\\
Production + Importations = Consommations intermédiaires + Demande finale + Exportations
}, il faudrait donc réduire les importations du montant du déficit et augmenter la valeur ajoutée de la production industrielle d’autant\footnote{Lorsqu’on augmente la production industrielle, les consommations intermédiaires de l’industrie augmentent aussi. La valeur ajoutée représente la différence~:\\
Valeur ajoutée = Production -  Consommations intermédiaires.\\
Donc~:\\
Valeur ajoutée + Importations = Demande finale + Exportations\\
À demande finale et exportations inchangées, les économies d’importation doivent être compensées par un accroissement de valeur ajoutée.
}.

Les ordres de grandeur peuvent être évalués en considérant que les ratios économiques actuels s’appliqueraient aux nouveaux investissements. Si l’on suppose que l’obtention de toute valeur ajoutée supplémentaire nécessite un accroissement des capacités de production, le montant des investissements à consentir s’évalue – en ordre de grandeur – comme le produit de cette valeur ajoutée par le ratio capital fixe / valeur ajoutée. On peut néanmoins supposer qu’une partie de la production supplémentaire sur le sol national puisse résulter d’un accroissement du taux d’utilisation des capacités productives en place. Les montants des investissements évalués dans le tableau ci-dessous apparaissent donc comme des limites supérieures.

Concernant l’évaluation de l’emploi créé – toujours en ordre de grandeur – on l’obtient en faisant le produit de la valeur ajoutée supplémentaire par le ratio actuel emploi / valeur ajoutée. On considère qu’il n’y a pas actuellement d’emplois inoccupés, si bien que toute production supplémentaire serait créatrice d’emplois, qu’elle soit assurée par les capacités productives existantes ou des investissements nouveaux. Il s’agit des emplois directs dans l’ensemble de l’industrie, sans compter les emplois induits dans les autres secteurs de l’économie.


\subsubsection{Caractéristiques de l’industrie manufacturière en 2017}
une table

une autre table

\subsubsection{Ordres de grandeur des moyens nécessaires à mobiliser}
une table

\subsubsection{Remarques}
\begin{itemize}
\item L’estimation pourrait devenir moins grossière si elle répondait à des objectifs de production déclinés par branche industrielle. On pourrait alors produire des évaluations d’investissements et d’emplois par branche en utilisant les ratios techniques qui leur sont spécifiques.

\item L’estimation ne prend pas en compte le fait que pour investir de nouvelles capacités productives, il faut produire en amont dans la branche des biens d’équipement, qui elle-même demande des produits intermédiaires auprès des autres branches. En première analyse, il faudrait donc prendre ne compte un certain effet multiplicateur sur les besoins en investissement (de l’ordre de 20 à 30\%). Mais ce raisonnement a toutes les chances d’être irréaliste parce que l’économie réelle ne se réduit pas à la mobilisation de moyens financiers~: il faut aussi mobiliser des capacités productives et des compétences humaines. Il y a tout lieu de penser que pour amorcer un redressement industriel, des importations de biens d’équipement seront nécessaires, ce qui peut apparaître contradictoire avec l’objectif d’en réduire le volume, mais qui devrait n’être que temporaire.
\end{itemize}

\subsection{Une planification pour atteindre l’objectif d’équilibre industriel ?}

\textbf{Quelques idées à amender et développer}
Les montants financiers à investir sont importants. Ils sont à la mesure de dizaines d’années de déficit d’investissement dans l’industrie nationale. On ne peut envisager d’atteindre l’objectif d’un commerce extérieur équilibré en produits industriels qu’avec un plan d’action pluriannuel. 

La reconstitution d’une industrie nationale capable de répondre aux besoins du pays doit viser à la fois l’amélioration de la couverture des besoins nationaux, et la compétitivité à l’export. il n'y a pas en effet de position industrielle forte sans export. A l’inverse, il n'y a pas dans la durée de position forte à l'export, que ce soit dans l'industrie ou dans les services aux entreprises, sans position forte sur le marché intérieur.

Quel pourrait être la durée d’un tel plan ? La réflexion peut partir de l’évaluation des capacités financières qui ont été consacrées aux délocalisations industrielles par l’économie nationale. Les investissements directs français à l’étranger ont dépassé les investissements directs étrangers en France de 32 milliards par an en moyenne durant au moins 20 ans. En consacrant ces mêmes 32 milliards à l’industrie nationale, le plan de retour à l’équilibre industriel durerait une dizaine d’années. Il pourrait être raccourci par d’autres moyens financiers.

Un tel plan se confronterait immédiatement à la règle européenne de libre circulation des capitaux. Il pose donc la question des moyens qu’aurait la France de s’en affranchir ou la contourner.

Un tel plan nécessite aussi de mobiliser bien d’autres moyens que les moyens financiers~:
\begin{itemize}
\item il ne suffit pas de raisonner en termes de créations d’emplois, il faut aussi s’assurer de la disponibilité des compétences humaines professionnelles. Cela pose la question des formations et de la priorité à donner aux métiers de l’industrie. 
\item il ne suffit pas de raisonner en termes d’augmentation-substitution de productions, il faut aussi s’assurer de la capacité à ce que les nouvelles productions trouvent des débouchés sur le marché. Cela pose la question de l’intervention de l’Etat en matière de subventions et de marché publics, autant de moyens d’action contraires aux dispositions des traités européens.
\end{itemize}

Un tel plan nécessiterait d’être décliné en objectifs concrets et cohérents entre eux (besoins finals et intermédiaires à satisfaire, investissements, financements, formations, débouchés…).  Il doit être pragmatique. Par exemple, certaines ressources nécessaires à un plan d’équilibre industriel peuvent être introuvables en France et doivent être importées. Le plan d’équilibre n’étant pas un plan d’autarcie, il conviendrait évidemment d’utiliser ce levier tout en évitant que cela compromette sa réalisation…

Un tel plan serait irréalisable sans que l’Etat ne se dote d’une organisation pour en décliner les objectifs, les moyens et le suivi de sa réalisation. Aujourd’hui, l’étude de la politique de l’industrie relève d’un service d’une direction du Ministère des Finances (la DGE, Direction générale des entreprises). Cela résulte de politiques délibérées de privation des moyens d’intervention publique dans le domaine de l’industrie, liée à l’idéologie libérale du marché roi. Il est significatif que la mission officielle assignée à la DGE en matière de “développement de nouveaux secteurs” pour “développer la compétitivité et la croissance des entreprises de l’industrie et des services” ne cite que “les services aux entreprises et à la personne”. Son action se réduit à : “analyser les meilleures pratiques internationales, écouter les acteurs économiques pour être une force de propositions des ministres dans tous les domaines de la compétitivité des entreprises.” Une seule chose est planifiée : l’absence totale d’ambition industrielle et l’interdiction d’en manifester.

Plus de 20 ans de renoncements par l’Etat à une politique industrielle digne de ce nom ont fini par affaiblir considérablement sa base de compétences et ses divers outils d’intervention, dont au premier chef la DGE, mais aussi nombre de services techniques dépendant des différents ministères concernés. Un audit des compétences encore disponibles devra être réalisé d’urgence, et mis en regard des besoins.

Il sera nécessaire de reconstituer un véritable service public d’études des besoins nationaux en produits industriels et contrôles stratégiques de productions, des objectifs de développement et des moyens associés.





\end{document}
